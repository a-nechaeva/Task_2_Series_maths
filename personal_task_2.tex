\documentclass[a5paper, 10pt]{article}

% Текст
\usepackage[utf8]{inputenc} % UTF-8 кодировка
\usepackage[russian]{babel} % Русский язык
\usepackage{indentfirst} % красная строка в первом параграфе в главе
% Отображение страниц
\usepackage{geometry} % размеры листа и отступов
\geometry{
	left=12mm,
	top=25mm,
	right=15mm,
	bottom=17mm,
	marginparsep=0mm,
	marginparwidth=0mm,
	headheight=10mm,
	headsep=7mm,
	nofoot}
\usepackage{afterpage,fancyhdr} % настройка колонтитулов
\pagestyle{fancy}
\fancypagestyle{style}{ % создание нового стиля style
	\fancyhf{} % очистка колонтитулов
	\fancyhead[LO, RE]{Сходимость рядов} % название документа наверху
	\fancyhead[RO, LE]{\leftmark} % название section наверху
	\fancyfoot[RO, LE]{\thepage} % номер страницы справа внизу на нечетных и слева внизу на четных
	\renewcommand{\headrulewidth}{0.25pt} % толщина линии сверху
	\renewcommand{\footrulewidth}{0pt} % толцина линии снизу
}
\fancypagestyle{plain}{ % создание нового стиля plain -- полностью пустого
	\fancyhf{}
	\renewcommand{\headrulewidth}{0pt}
}
\fancypagestyle{title}{ % создание нового стиля title -- для титульной страницы
	\fancyhf{}
	\fancyhead[C]{{\footnotesize
			Министерство образования и науки Российской Федерации\\
			Федеральное государственное автономное образовательное учреждение высшего образования
	}}
	\fancyfoot[C]{{\large 
			Санкт-Петербург, 2023
	}}
	\renewcommand{\headrulewidth}{0pt}
}

% Математика
\usepackage{amsmath, amsfonts, amssymb, amsthm} % Набор пакетов для математических текстов
%\usepackage{dmvnbase} % мехматовский пакет latex-сокращений
\usepackage{cancel} % зачеркивание для сокращений
% Рисунки и фигуры
\usepackage[pdftex]{graphicx} % вставка рисунков
\usepackage{wrapfig, subcaption} % вставка фигур, обтекая текст
\usepackage{caption} % для настройки подписей
\captionsetup{figurewithin=none,labelsep=period, font={small,it}} % настройка подписей к рисункам
% Рисование
\usepackage{tikz} % рисование
\usepackage{circuitikz}
\usepackage{pgfplots} % графики
% Таблицы
\usepackage{multirow} % объединение строк
\usepackage{multicol} % объединение столбцов
% Остальное
\usepackage[unicode, pdftex]{hyperref} % гиперссылки
\usepackage{enumitem} % нормальное оформление списков
\setlist{itemsep=0.15cm,topsep=0.15cm,parsep=1pt} % настройки списков
% Теоремы, леммы, определения...
\theoremstyle{definition}
\newtheorem{Def}{Определение}
\newtheorem*{Axiom}{Аксиома}
\theoremstyle{plain}
\newtheorem{Th}{Теорема}
\newtheorem{Lem}{Лемма}
\newtheorem{Cor}{Следствие}
\newtheorem{Ex}{Пример}
\theoremstyle{remark}
\newtheorem*{Note}{Замечание}
\newtheorem*{Solution}{Решение}
\newtheorem*{Proof}{Доказательство}
% Свои команды
\newcommand{\comb}[1]{\left[\hspace{-4pt}\begin{array}{l}#1\end{array}\right.\hspace{-5pt} } % совокупность уравнений
% Титульный лист
\usepackage{csvsimple-l3}
\newcommand*{\titlePage}{
	\thispagestyle{title}
	\begingroup
	\begin{center}
		%		{\footnotesize
			%			Министерство образования и науки Российской Федерации\\
			%			Федеральное государственное автономное образовательное учреждение высшего образования
			%		}
		%		
		\vspace*{6ex}
		
		{\small
			САНКТ-ПЕТЕРБУРГСКИЙ НАЦИОНАЛЬНЫЙ ИССЛЕДОВАТЕЛЬСКИЙ УНИВЕРСИТЕТ ИТМО
		}
		
		\vspace*{2ex}
		
		{\normalsize
			Факультет систем управления и робототехники
		}
		
		\vspace*{15ex}
		
		{\Large \bfseries 
			Индивидуальное задание № 2
		}

\vspace*{2ex}
		
		{\normalsize
			по дисциплине "Математический анализ"
		}
\vspace*{5ex}
		
		{\Large \bfseries 
			 Вариант № 20
		}
	\end{center}
	\vspace*{20ex}
	\begin{flushright}
		{\large 
			\underline{Выполнила}: студентка гр. \textbf{R3138}\\
			\begin{flushright}
				\textbf{Нечаева А. А.}\\
			\end{flushright}
		}
		
		\vspace*{5ex}
		
		{\large 
			\underline{Преподаватель}: \textit{<ФИО ПРЕПОДАВАТЕЛЯ>}
		}
	\end{flushright}	
	\newpage
	\setcounter{page}{2}
	\endgroup}

\begin{document}
	\titlePage
	\pagestyle{style}
\newpage
\section{ Сходимость числовых рядов}
Исследовать ряды на сходимость. Для знакопеременных рядов исследовать абсолютную и условную сходимость.	

\subsection{a}
\begin{equation}
\sum \limits_{n = 1}^{\infty} \frac{\left(2n\right)!!}{n^{n+\frac{3}{2}}} \cdot ln \frac{2^n+1}{2^n}
\end{equation}
1. Ряд знакопостоянный\\
2. Воспрользуемся признаком д'Аламбера в предельной форме
\begin{multline*}
\rho = \lim_{n \to \infty} \frac{a_{n+1}}{a_n} = \lim_{n \to \infty} \frac{\left(2n+2\right)!! \cdot ln\frac{2^{n+1}+1}{2^{n+1}}\cdot n^{n+\frac{3}{2}}}{\left(2n\right)!! \cdot ln\frac{2^{n}+1}{2^{n}}\cdot (n+1)^{n + 1+\frac{3}{2}}} =\\
=  \lim_{n \to \infty} \frac{\left(2n\right)!!\cdot \left(2n+2\right) \cdot ln\left(1 + \frac{1}{2^{n+1}}\right)\cdot n^{n+\frac{3}{2}}}{\left(2n\right)!! \cdot (n+1)\cdot ln\left(1 + \frac{1}{2^{n}}\right)\cdot (n+1)^{n + \frac{3}{2}}} =
 \lim_{n \to \infty} \frac{ 2\cdot  \frac{1}{2^{n+1}}\cdot n^{n+\frac{3}{2}}}{ \frac{1}{2^{n}}\cdot (n+1)^{n + \frac{3}{2}}}=\\ =\lim_{n \to \infty} \left(\frac{n}{n+1} \right)^{n+\frac{3}{2}}= \lim_{n \to \infty} \left(1 - \frac{1}{n+1} \right)^{n+\frac{3}{2}}=\lim_{n \to \infty}  \left( \left(1 - \frac{1}{n+1} \right)^{-(n+1) } \right)^{-\frac{n+\frac{3}{2}}{n+1} }=\\
= \frac{1}{e} < 1
\end{multline*}
\underline{Ответ:} ряд  \textbf{сходится}  по признаку  д'Аламбера


\subsection{б}
\begin{equation}
\sum \limits_{n = 1}^{\infty} \left( arctg \frac{n+1}{n^2} - ln \left(1+ tg \frac{1}{n} \right) \right)^2
\end{equation}
\\
1. Ряд знакопостоянный\\
2. Воспрользуемся разложением функций в ряд Маклорена при $n \to \infty$
\begin{gather*}
arctg \frac {n+1}{n^2} = \frac {1}{n} + \frac {1}{n^2}  + o\left( \frac{1}{n^3} \right) \\
tg \frac{1}{n} =  \frac {1}{n}  + o\left( \frac{1}{n^3} \right)\\
ln \left(1 +  \frac {1}{n}  + o\left( \frac{1}{n^3} \right)\right) =  \frac {1}{n} -  \frac{1}{2n^2}  + o\left( \frac{1}{n^3} \right)
\end{gather*}
Тогда подставляя получаем:
\begin{multline}
\left(  \frac {1}{n} + \frac {1}{n^2}  + O\left( \frac{1}{n^3} \right) - \left(  \frac {1}{n} -  \frac{1}{2n^2}  + O\left( \frac{1}{n^3} \right)\right) \right)^2 = \left(  \frac {3}{2n^2}  + O\left( \frac{1}{n^3} \right) \right)^2 =\\
= \frac{9}{4n^4} +  \frac{3}{n^2} \cdot  O\left( \frac{1}{n^3} \right)+ O\left( \frac{1}{n^6} \right)
\end{multline}
В силу сходимости рядов $\frac{9}{4n^4} $, $\frac{3}{n^2} \cdot  o\left( \frac{1}{n^3} \right)$ и $o\left( \frac{1}{n^6} \right)$ и согласно арифметическим свойствам рядов их сумма тоже сходится.\\

\underline{Ответ:} ряд  \textbf{сходится}

\subsection{в}
\begin{equation}
\sum \limits_{n = 1}^{\infty} \frac{\left(n+1\right) \cos 2n}{n^2-\ln n}
\end{equation}
\\
1. Ряд знакопеременный\\
Заметим, что перемена знака вызвано только $\cos 2n$, так как $n^2 > \ln n$\\
2. Исследуем его на \textit{абсолютную сходимость} \\
3. Исследуем его на \textit{условную сходимость} \\
\\\\
\\\\
\subsection{г}
\begin{equation}
\sum \limits_{n = 1}^{\infty} \left(\arccos \left(1-\frac{1}{n} \right) \right)^\frac{1}{n}
\end{equation}
\\
1. Ряд знакопостоянный\\
2. Проверим, выполнен ли \textit{необходимый} признак сходимости числового ряда $ \lim_{n \to \infty} a_n = 0$
\begin{multline*}
 \lim_{n \to \infty} \left(\arccos \left(1-\frac{1}{n} \right) \right)^\frac{1}{n} = \left| \left| 0 \leq1-\frac{1}{n} \leq 1 \right| \right| =\\
= \lim_{n \to \infty}  \left(\arcsin \sqrt{1 - \left(1-\frac{1}{n} \right)^2} \right)^\frac{1}{n}=  \lim_{n \to \infty}  \left(\arcsin \sqrt{\frac{2}{n} -  \frac{1}{n^2}} \right)^\frac{1}{n} =\\ =  \lim_{n \to \infty}  \left(\sqrt{\frac{2}{n} + O\frac{1}{n^2}} \right)^\frac{1}{n} = \lim_{n \to \infty} \left(\sqrt{\frac{2}{n}} \right)^\frac{1}{n}= \lim_{n \to \infty} \left(\frac{2}{n} \right)^\frac{1}{2n}= \lim_{n \to \infty} e^{\frac{1}{2n} \cdot \ln \frac{2}{n}} = \\ =  e^{ \lim_{n \to \infty}\frac{1}{2n} \cdot \ln \frac{2}{n}}
\end{multline*}
Далее вычислим предел степени экспоненты, применив теорему Штольца
\begin{multline*}
\lim_{n \to \infty}\frac{1}{2n} \ln \frac{2}{n} =  \lim_{n \to \infty}\frac{ \ln \frac{2}{n} -  \ln \frac{2}{n - 1}}{2n - 2n + 2} = \lim_{n \to \infty}\frac{ \ln \frac{n -1}{n}}{2} =  \lim_{n \to \infty}\frac{ \ln \left( 1 - \frac{1}{n}\right)}{2} =  \lim_{n \to \infty}\frac{- \frac{1}{n}}{2} = 0\\
\end{multline*}
Подставляя вычисленное значение, получим
\begin{equation}
 \lim_{n \to \infty} \left(\arccos \left(1-\frac{1}{n} \right) \right)^\frac{1}{n} = e^0 = 1 \neq 0 \\
\end{equation}
Следовательно, ряд расходится.\\\\
\underline{Ответ:} ряд  \textbf{расходится}

\subsection{д}
\begin{equation}
\sum \limits_{n = 1}^{\infty} e^n \left( \frac{n}{n+1} \right)^{n^2+3n}
\end{equation}
\\
1. Ряд знакопостоянный\\
2. Проверим, выполнен ли \textit{необходимый} признак сходимости числового ряда $ \lim_{n \to \infty} a_n = 0$
\begin{multline*}
\lim_{n \to \infty} e^n \left( \frac{n}{n+1} \right)^{n^2+3n} =
 \lim_{n \to \infty} e^n \left( 1-\frac{1}{n+1} \right)^{-(n+1) \cdot \frac{n^2+3n}{-n-1}} = \\ =
\lim_{n \to \infty} e^n\cdot e^{ \frac{n^2+3n}{-n-1}}  = \lim_{n \to \infty} e^n\cdot e^{-n} = 1 \neq 0 \\
\end{multline*}
Следовательно, ряд расходится.\\\\
\underline{Ответ:} ряд  \textbf{расходится}
\subsection{e*}
\begin{equation}
\sum \limits_{n = 1}^{\infty}\int_0^1 x \cdot \cos nx \, dx
\end{equation}
\\
\\
\section{Область сходимости функционального ряда}
\begin{equation}
\sum \limits_{n = 1}^{\infty}\frac{\cos( nx) \cdot \cos (3nx) }{n^{\frac{x}{3}}}
\end{equation}
1. Нахождение области абсолютной сходимости ряда\\
Заметим, что все члены ряда определены для всех вещественных значений $x$, причем в точках  $x=\frac{\pi k}{2}, \,\, k \in Z$ все члены ряда равны нулю, поэьлму в этих точках ряд сходится (абсолютно).\\
Преобразуем числитель в сумму
\begin{equation}
\cos( nx) \cdot \cos (3nx) = \frac{1}{2}\left( \cos(2nx) + \cos(4nx) \right)
\end{equation}
Тогда
\begin{equation}
\frac{\cos( 2nx) }{n^{\frac{x}{3}}}+\frac{ \cos (4nx) }{n^{\frac{x}{3}}}
\end{equation}
\end{document}













