\documentclass[a5paper, 10pt]{article}

% Текст
\usepackage[utf8]{inputenc} % UTF-8 кодировка
\usepackage[russian]{babel} % Русский язык
\usepackage{indentfirst} % красная строка в первом параграфе в главе
% Отображение страниц
\usepackage{geometry} % размеры листа и отступов
\geometry{
	left=12mm,
	top=25mm,
	right=15mm,
	bottom=17mm,
	marginparsep=0mm,
	marginparwidth=0mm,
	headheight=10mm,
	headsep=7mm,
	nofoot}
\usepackage{afterpage,fancyhdr} % настройка колонтитулов
\pagestyle{fancy}
\fancypagestyle{style}{ % создание нового стиля style
	\fancyhf{} % очистка колонтитулов
	\fancyhead[LO, RE]{} % название документа наверху
	\fancyhead[RO, LE]{\leftmark} % название section наверху
	\fancyfoot[RO, LE]{\thepage} % номер страницы справа внизу на нечетных и слева внизу на четных
	\renewcommand{\headrulewidth}{0.25pt} % толщина линии сверху
	\renewcommand{\footrulewidth}{0pt} % толцина линии снизу
}
\fancypagestyle{plain}{ % создание нового стиля plain -- полностью пустого
	\fancyhf{}
	\renewcommand{\headrulewidth}{0pt}
}
\fancypagestyle{title}{ % создание нового стиля title -- для титульной страницы
	\fancyhf{}
	\fancyhead[C]{{\footnotesize
			Министерство образования и науки Российской Федерации\\
			Федеральное государственное автономное образовательное учреждение высшего образования
	}}
	\fancyfoot[C]{{\large 
			Санкт-Петербург, 2023
	}}
	\renewcommand{\headrulewidth}{0pt}
}

% Математика
\usepackage{amsmath, amsfonts, amssymb, amsthm} % Набор пакетов для математических текстов
%\usepackage{dmvnbase} % мехматовский пакет latex-сокращений
\usepackage{cancel} % зачеркивание для сокращений
% Рисунки и фигуры
\usepackage[pdftex]{graphicx} % вставка рисунков
\usepackage{wrapfig, subcaption} % вставка фигур, обтекая текст
\usepackage{caption} % для настройки подписей
\captionsetup{figurewithin=none,labelsep=period, font={small,it}} % настройка подписей к рисункам
% Рисование
\usepackage{tikz} % рисование
\usepackage{circuitikz}
\usepackage{pgfplots} % графики
% Таблицы
\usepackage{multirow} % объединение строк
\usepackage{multicol} % объединение столбцов
% Остальное
\usepackage[unicode, pdftex]{hyperref} % гиперссылки
\usepackage{enumitem} % нормальное оформление списков
\setlist{itemsep=0.15cm,topsep=0.15cm,parsep=1pt} % настройки списков
% Теоремы, леммы, определения...
\theoremstyle{definition}
\newtheorem{Def}{Определение}
\newtheorem*{Axiom}{Аксиома}
\theoremstyle{plain}
\newtheorem{Th}{Теорема}
\newtheorem{Lem}{Лемма}
\newtheorem{Cor}{Следствие}
\newtheorem{Ex}{Пример}
\theoremstyle{remark}
\newtheorem*{Note}{Замечание}
\newtheorem*{Solution}{Решение}
\newtheorem*{Proof}{Доказательство}
% Свои команды
\newcommand{\comb}[1]{\left[\hspace{-4pt}\begin{array}{l}#1\end{array}\right.\hspace{-5pt} } % совокупность уравнений
% Титульный лист
\usepackage{csvsimple-l3}
\newcommand*{\titlePage}{
	\thispagestyle{title}
	\begingroup
	\begin{center}
		%		{\footnotesize
			%			Министерство образования и науки Российской Федерации\\
			%			Федеральное государственное автономное образовательное учреждение высшего образования
			%		}
		%		
		\vspace*{6ex}
		
		{\small
			САНКТ-ПЕТЕРБУРГСКИЙ НАЦИОНАЛЬНЫЙ ИССЛЕДОВАТЕЛЬСКИЙ УНИВЕРСИТЕТ ИТМО
		}
		
		\vspace*{2ex}
		
		{\normalsize
			Факультет систем управления и робототехники
		}
		
		\vspace*{15ex}
		
		{\Large \bfseries 
			Индивидуальное задание № 2
		}

\vspace*{2ex}
		
		{\normalsize
			по дисциплине "Математический анализ"
		}
\vspace*{5ex}
		
		{\Large \bfseries 
			 Вариант № 20
		}
	\end{center}
	\vspace*{20ex}
	\begin{flushright}
		{\large 
			\underline{Выполнила}: студентка гр. \textbf{R3138}\\
			\begin{flushright}
				\textbf{Нечаева А. А.}\\
			\end{flushright}
		}
		
		\vspace*{5ex}
		
		{\large 
			\underline{Преподаватель}: \textit{<ФИО ПРЕПОДАВАТЕЛЯ>}
		}
	\end{flushright}	
	\newpage
	\setcounter{page}{2}
	\endgroup}

\begin{document}
	\titlePage
	\pagestyle{style}
\newpage
\section{ Сходимость числовых рядов}
Исследовать ряды на сходимость. Для знакопеременных рядов исследовать абсолютную и условную сходимость.	

\subsection{a}
\begin{equation}
\sum \limits_{n = 1}^{\infty} \frac{\left(2n\right)!!}{n^{n+\frac{3}{2}}} \cdot ln \frac{2^n+1}{2^n}
\end{equation}
1. Ряд знакопостоянный\\
2. Воспрользуемся признаком д'Аламбера в предельной форме
\begin{multline*}
\rho = \lim_{n \to \infty} \frac{a_{n+1}}{a_n} = \lim_{n \to \infty} \frac{\left(2n+2\right)!! \cdot ln\frac{2^{n+1}+1}{2^{n+1}}\cdot n^{n+\frac{3}{2}}}{\left(2n\right)!! \cdot ln\frac{2^{n}+1}{2^{n}}\cdot (n+1)^{n + 1+\frac{3}{2}}} =\\
=  \lim_{n \to \infty} \frac{\left(2n\right)!!\cdot \left(2n+2\right) \cdot ln\left(1 + \frac{1}{2^{n+1}}\right)\cdot n^{n+\frac{3}{2}}}{\left(2n\right)!! \cdot (n+1)\cdot ln\left(1 + \frac{1}{2^{n}}\right)\cdot (n+1)^{n + \frac{3}{2}}} =
 \lim_{n \to \infty} \frac{ 2\cdot  \frac{1}{2^{n+1}}\cdot n^{n+\frac{3}{2}}}{ \frac{1}{2^{n}}\cdot (n+1)^{n + \frac{3}{2}}}=\\ =\lim_{n \to \infty} \left(\frac{n}{n+1} \right)^{n+\frac{3}{2}}= \lim_{n \to \infty} \left(1 - \frac{1}{n+1} \right)^{n+\frac{3}{2}}=\lim_{n \to \infty}  \left( \left(1 - \frac{1}{n+1} \right)^{-(n+1) } \right)^{-\frac{n+\frac{3}{2}}{n+1} }=\\
= \frac{1}{e} < 1
\end{multline*}
\underline{Ответ:} ряд  \textbf{сходится}  по признаку  д'Аламбера


\newpage
\subsection{б}
\begin{equation}
\sum \limits_{n = 1}^{\infty} \left( arctg \frac{n+1}{n^2} - ln \left(1+ tg \frac{1}{n} \right) \right)^2
\end{equation}
\\
1. Ряд знакопостоянный\\
2. Воспрользуемся разложением функций в ряд Маклорена при $n \to \infty$
\begin{gather*}
arctg \frac {n+1}{n^2} = \frac {1}{n} + \frac {1}{n^2}  + o\left( \frac{1}{n^3} \right) \\
tg \frac{1}{n} =  \frac {1}{n}  + o\left( \frac{1}{n^3} \right)\\
ln \left(1 +  \frac {1}{n}  + o\left( \frac{1}{n^3} \right)\right) =  \frac {1}{n} -  \frac{1}{2n^2}  + o\left( \frac{1}{n^3} \right)
\end{gather*}
Тогда подставляя получаем:
\begin{multline}
\left(  \frac {1}{n} + \frac {1}{n^2}  + O\left( \frac{1}{n^3} \right) - \left(  \frac {1}{n} -  \frac{1}{2n^2}  + O\left( \frac{1}{n^3} \right)\right) \right)^2 = \left(  \frac {3}{2n^2}  + O\left( \frac{1}{n^3} \right) \right)^2 =\\
= \frac{9}{4n^4} +  \frac{3}{n^2} \cdot  O\left( \frac{1}{n^3} \right)+ O\left( \frac{1}{n^6} \right)
\end{multline}
В силу сходимости рядов $\frac{9}{4n^4} $, $\frac{3}{n^2} \cdot  o\left( \frac{1}{n^3} \right)$ и $o\left( \frac{1}{n^6} \right)$ и согласно арифметическим свойствам рядов их сумма тоже сходится.\\

\underline{Ответ:} ряд  \textbf{сходится}


\newpage
\subsection{в}
\begin{equation}
\sum \limits_{n = 1}^{\infty} \frac{\left(n+1\right) \cos 2n}{n^2-\ln n}
\end{equation}
\\
1. Ряд знакопеременный\\
Заметим, что перемена знака вызвано только $\cos 2n$, так как $n^2 > \ln n$.\\
2. Исследуем его на \textit{абсолютную сходимость}. \\
\begin{equation}
\frac{\left(n+1\right) \left| \cos 2n \right|}{n^2-\ln n} \geq \frac{\left(n+1\right) \left( \cos 2n \right)^2}{n^2-\ln n}=
\frac{\left(n+1\right)}{n^2-\ln n} - \frac{\left(n+1\right) \left( \sin 2n \right)^2}{n^2-\ln n}
\end{equation}
Отдельно рассмотрим ряд $\sum \limits_{n = 1}^{\infty} \frac{\left(n+1\right)}{n^2-\ln n}$
\begin{equation}
\frac{\left(n+1\right)}{n^2-\ln n} \geq \frac{\left(n+1\right)}{n^2+ n}=  \frac{1}{n}
\end{equation}
По признаку сравнения ряд $\sum \limits_{n = 1}^{\infty} \frac{\left(n+1\right)}{n^2-\ln n}$ расходится, следовательно, $\sum \limits_{n = 1}^{\infty} \frac{\left(n+1\right) \left| \cos 2n \right|}{n^2-\ln n}$ также расходится. \textbf{Абсолютной сходимости нет.}\\
3. Исследуем его на \textit{условную сходимость} Рассмотрим соотвествующий несобственный интеграл. \\
\begin{equation}
\int_{1}^{\infty}\frac{\left(x+1\right)\cos 2x}{x^2-\ln x} \, dx
\end{equation}
Обозначим $f(x) =\cos 2x $ и $g(x) =\frac{(x+1)}{x^2-\ln x}$. Первообразная $F(x) = \frac{1}{2} \sin 2x$ ограничена и непрерывна на промежутке интегрирования. Установим монотонность убывания функции $g(x)$ для этого вычислим производную.
\begin{equation}
g'(x) = \frac{x^2-\ln x - \left( 2x - \frac{1}{x}\right) (x+1)}{\left( x^2-\ln x \right)^2 }= - \frac{x^2+\ln x-1-\frac{1}{x} + 2x}{\left( x^2-\ln x \right)^2}
\end{equation}
На промежутке $[1, \infty)$ выполнено неравенство $g'(x) < 0$ и\\ $\lim_{x \to \infty}\frac{(x+1)}{x^2-\ln x} = 0 $. Значит, по признаку Дирихле интеграл $\int_{1}^{\infty}\frac{\left(x+1\right)\cos 2x}{x^2-\ln x} \, dx$ сходится условно, вместе с ним условно сходится и ряд $\sum \limits_{n = 1}^{\infty} \frac{\left(n+1\right) \cos 2n}{n^2-\ln n}$.
\\\\
\underline{Ответ:} ряд  \textbf{сходится условно}


\newpage
\subsection{г}
\begin{equation}
\sum \limits_{n = 1}^{\infty} \left(\arccos \left(1-\frac{1}{n} \right) \right)^\frac{1}{n}
\end{equation}
\\
1. Ряд знакопостоянный\\
2. Проверим, выполнен ли \textit{необходимый} признак сходимости числового ряда $ \lim_{n \to \infty} a_n = 0$
\begin{multline*}
 \lim_{n \to \infty} \left(\arccos \left(1-\frac{1}{n} \right) \right)^\frac{1}{n} = \left| \left| 0 \leq1-\frac{1}{n} \leq 1 \right| \right| =\\
= \lim_{n \to \infty}  \left(\arcsin \sqrt{1 - \left(1-\frac{1}{n} \right)^2} \right)^\frac{1}{n}=  \lim_{n \to \infty}  \left(\arcsin \sqrt{\frac{2}{n} -  \frac{1}{n^2}} \right)^\frac{1}{n} =\\ =  \lim_{n \to \infty}  \left(\sqrt{\frac{2}{n} + O\frac{1}{n^2}} \right)^\frac{1}{n} = \lim_{n \to \infty} \left(\sqrt{\frac{2}{n}} \right)^\frac{1}{n}= \lim_{n \to \infty} \left(\frac{2}{n} \right)^\frac{1}{2n}= \lim_{n \to \infty} e^{\frac{1}{2n} \cdot \ln \frac{2}{n}} = \\ =  e^{ \lim_{n \to \infty}\frac{1}{2n} \cdot \ln \frac{2}{n}}
\end{multline*}
Далее вычислим предел степени экспоненты, применив теорему Штольца
\begin{multline*}
\lim_{n \to \infty}\frac{1}{2n} \ln \frac{2}{n} =  \lim_{n \to \infty}\frac{ \ln \frac{2}{n} -  \ln \frac{2}{n - 1}}{2n - 2n + 2} = \lim_{n \to \infty}\frac{ \ln \frac{n -1}{n}}{2} =  \lim_{n \to \infty}\frac{ \ln \left( 1 - \frac{1}{n}\right)}{2} =  \lim_{n \to \infty}\frac{- \frac{1}{n}}{2} = 0\\
\end{multline*}
Подставляя вычисленное значение, получим
\begin{equation}
 \lim_{n \to \infty} \left(\arccos \left(1-\frac{1}{n} \right) \right)^\frac{1}{n} = e^0 = 1 \neq 0 \\
\end{equation}
Следовательно, ряд расходится.\\\\
\underline{Ответ:} ряд  \textbf{расходится}


\newpage
\subsection{д}
\begin{equation}
\sum \limits_{n = 1}^{\infty} e^n \left( \frac{n}{n+1} \right)^{n^2+3n}
\end{equation}
\\
1. Ряд знакопостоянный\\
2. Проверим, выполнен ли \textit{необходимый} признак сходимости числового ряда $ \lim_{n \to \infty} a_n = 0$
\begin{multline*}
\lim_{n \to \infty} e^n \left( \frac{n}{n+1} \right)^{n^2+3n} =
 \lim_{n \to \infty} e^n \left( 1-\frac{1}{n+1} \right)^{-(n+1) \cdot \frac{n^2+3n}{-n-1}} = \\ =
\lim_{n \to \infty} e^n\cdot e^{ \frac{n^2+3n}{-n-1}}  = \lim_{n \to \infty} e^n\cdot e^{-n} = 1 \neq 0 \\
\end{multline*}
Следовательно, ряд расходится.\\\\
\underline{Ответ:} ряд  \textbf{расходится}
\newpage
\subsection{e*}
\begin{equation}
\sum \limits_{n = 1}^{\infty}\int_0^1 x \cdot \cos nx \, dx
\end{equation}


\newpage
\section{Область сходимости функционального ряда}
\begin{equation}
\sum \limits_{n = 1}^{\infty}\frac{\cos( nx) \cdot \cos (3nx) }{n^{\frac{x}{3}}}
\end{equation}
1. Нахождение области абсолютной сходимости ряда\\

Заметим, что все члены ряда определены для всех вещественных значений $x$, причем при $\cos(nx) = 0$  и $\cos(3 nx) = 0$  все члены ряда равны нулю, поэтому \hypertarget{pdf}{в точках}  $x = \frac{\pi}{2} \cdot (2k + 1), \,\, k \in Z$  и $x = \frac{\pi}{6} \cdot (2m + 1), \,\, m \in Z$ ряд сходится (абсолютно).\\


Преобразуем числитель в сумму
\begin{equation*}
\cos( nx) \cdot \cos (3nx) = \frac{1}{2}\left( \cos(2nx) + \cos(4nx) \right)
\end{equation*}
Тогда
\begin{equation*}
\frac{\cos( nx) \cdot \cos (3nx) }{n^{\frac{x}{3}}} = \frac{\cos( 2nx) }{2n^{\frac{x}{3}}}+\frac{ \cos (4nx) }{2n^{\frac{x}{3}}}
\end{equation*}
К каждому из полученных рядов применим признак Дирихле.\\
Суммы 
\begin{equation*}
\left| \sum \limits_{n = 1}^{\infty} \cos( 2nx) \right| = \left| \frac{\sin (n x) \cos((n+1)x)}{\sin x} \right| =
 \left| \frac{-\sin (x) +  \sin((2n+1)x)}{2\sin x} \right| \leq \frac{1}{ \left|\sin x\right|}
\end{equation*}
\begin{equation*}
\left| \sum \limits_{n = 1}^{\infty} \cos( 4nx) \right| = \left| \frac{-\sin (2x) +  \sin(2x(2n+1))}{2\sin 2x} \right| \leq \frac{1}{ \left|\sin 2x\right|}
\end{equation*}
ограничены при всех значениях $n$ для каждого фиксированого значения $x \neq \pi k, \, k \in Z$ и $x \neq \frac{\pi m}{2}, \, m \in Z$.\\
Заметим, что  последовательность $\frac{1}{n^{\frac{x}{3}}}$, монотонно убывает, стемясь к нулю, при $x > 0$. \\
Тогда исходный ряд сходится условно по признаку Дирихле\\
 при $x \in (0, + \infty) \backslash \left\{\pi k, \, \frac{\pi m}{2};  k, m \in N \right\} $. \\

Исследуем сходимость ряда в точках $x = \pi k, \, k \in N$.
\begin{equation*}
\sum \limits_{n = 1}^{\infty} \cos( 2n\cdot \pi k) = \sum \limits_{n = 1}^{\infty} 1 = \infty
\end{equation*}
\begin{equation*}
\sum \limits_{n = 1}^{\infty} \cos( 4n\cdot \pi k) = \sum \limits_{n = 1}^{\infty} 1 = \infty
\end{equation*}
Подставляя в исходную сумму ряда:
\begin{equation*}
 \frac{\cos( 2nx) }{2n^{\frac{x}{3}}}+\frac{ \cos (4nx) }{2n^{\frac{x}{3}}} = \frac{1 }{2n^{\frac{x}{3}}}+\frac{1 }{2n^{\frac{x}{3}}} = \frac{1 }{n^{\frac{x}{3}}}
\end{equation*}
Сходится при $n^{\frac{x}{3}} > n \to x > 3$, для других $x$ -- расходится.\\

Рассмотрим поведение ряда в точках $x = \frac{\pi m}{2}, \, m \in N$. В точках вида $x = \frac{\pi}{2} \cdot (2k + 1), \,\, k \in  Z$ ряд сходится абсолютно \hyperlink{pdf}{ (см. в начале решения)}, а сходимость ряда при четных значениях $m \in N$ доказана выше.\\\\
Область \textbf{условной} сходимости  $$x \in (0, + \infty)  \cup \left\{  \frac{\pi}{2} \cdot (2k + 1), \frac{\pi}{6} \cdot (2m + 1); \,\, k ,  m\in Z\right\}$$

Исследуем ряд на абсолютную сходимость в точках  $x \neq \frac{\pi}{2} \cdot (2k + 1), \,\, k \in Z$  и $x \neq \frac{\pi}{6} \cdot (2m + 1), \,\, m \in Z$.
\begin{multline*}
\left| \frac{\cos( nx) \cdot \cos (3nx) }{n^{\frac{x}{3}}}  \right| =  \frac{\left|\cos( nx) \cdot \cos (3nx)\right| }{n^{\frac{x}{3}}}
\geq   \frac{\left(\cos( nx) \cdot \cos (3nx)\right)^2 }{n^{\frac{x}{3}}} = \\
= \frac{\left(1 + \cos(2 nx)\right) \cdot \left(1 +  \cos (6nx)\right)}{4n^{\frac{x}{3}}} =
\frac{1}{4n^{\frac{x}{3}}}+\frac{ \cos(2 nx)}{4n^{\frac{x}{3}}}+\frac{\cos (6nx)}{4n^{\frac{x}{3}}}+\frac{\cos(2 nx) \cdot \cos (6nx)}{4n^{\frac{x}{3}}}
\end{multline*}
Сумма рядов выше расходится при $x \leq 3$, так как расходится ряд $\frac{1}{4n^{\frac{x}{3}}}$. Далее, будем рассматривать $x > 3$.
\begin{equation*}
\left| \frac{\cos( nx) \cdot \cos (3nx) }{n^{\frac{x}{3}}}  \right| \leq  \frac{1}{n^{\frac{x}{3}}}
\end{equation*}
Следовательно, при $x > 3$ ряд сходится \textbf{абсолютно}.\\
 
\underline{Ответ:} при $x \in (0, + \infty)  \cup \left\{  \frac{\pi}{2} \cdot (2k + 1), \frac{\pi}{6} \cdot (2m + 1); \,\, k ,  m\in Z\right\}$  ряд сходится \textbf{условно}, при $x \in (3, + \infty)  \cup \left\{  \frac{\pi}{2} \cdot (2k + 1), \frac{\pi}{6} \cdot (2m + 1); \,\, k ,  m\in Z\right\}$  -- \textbf{абсолютно}.


\newpage
\section{Равномерная сходимость функциональной последовательности}
\begin{equation*}
f_n (x) = n \ln \left( 1+ \frac{1}{nx} \right)
\end{equation*}

Найдем функцию $\phi(x)$, к которой сходится данная последовательность при фиксированном $x$ и $n \to \infty$:
\begin{equation}
\phi(x) = \lim_{n \to \infty} f_n (x) = \lim_{n \to \infty} n \ln \left( 1+ \frac{1}{nx} \right) =
\lim_{n \to \infty} n  \frac{1}{nx} = \frac{1}{x}
\end{equation}
а)  $x \in (0, 2)$. Докажем, что на этом промежутке последовательность сходится к предельной функции неравномерно. Последовательность функции $f_n (x)$ сходится к предельной функции неравномерно на промежутке $\langle a, b \rangle$, если существует $\epsilon_0 > 0$ такое, что какое бы значение $n_0 \in N$ мы ни взяли, можно найти значения $n \geq n_0, n \in N$ и $x_n \in \langle a, b \rangle$ такие, что $\left| f_n (x) - \phi(x)  \right| \geq \epsilon_0$. \\
Возьмем $x_n = \frac{1}{n}$ и $\epsilon_0 = 1$. Тогда
\begin{equation*}
\left| f_n (x_n) - \phi(x_n)  \right| = n \left( \frac{1}{nx_n}- \ln \left( 1+ \frac{1}{nx_n} \right)  \right) =  
 n \left( 1- \ln \left( 2 \right)  \right)
\end{equation*}
Так как последняя величина стремится к бесконечности, то, начиная с некоторого $n$, выполняется неравенство $ n \left( 1- \ln \left( 2 \right)  \right) > \epsilon_0 = 1$, следовательно, последовательность сходится к предельной функции \textbf{неравномерно}.\\

б)  $x \in (2, + \infty)$. Равномерная сходимость будет доказана, если мы докажем, что $\sup \limits_{x \in (2, + \infty)} \left| f_n (x) - \phi(x)  \right| $ стремится к нулю, при $n \to \infty$.
\begin{multline}
\sup \limits_{x \in (2, + \infty) } \left| f_n (x) - \phi(x)  \right| = \sup \limits_{x \in (2, + \infty) } \left| n \ln \left( 1+ \frac{1}{nx} \right) - \frac{1}{x}  \right| =\\ = \sup \limits_{x \in (2, + \infty) }  n \left| \ln \left( 1+ \frac{1}{nx} \right) - \frac{1}{nx} \right| 
= \sup \limits_{x \in (2, + \infty) }  n \left( \frac{1}{nx}- \ln \left( 1+ \frac{1}{nx} \right)  \right)
\end{multline}
Так как функция, стоящая под знаком супремума, дифференцируема на данном промежутке и, следовательно, непрерывна, то
\begin{equation*}
\sup \limits_{x \in (2, + \infty) } \left| f_n (x) - \phi(x)  \right| = \max \limits_{x \in (2, + \infty) }  n \left( \frac{1}{nx}- \ln \left( 1+ \frac{1}{nx} \right)  \right),
\end{equation*}
и этот максимум можно найти методами дифференциального исчисления:
\begin{equation*}
n \frac{d}{dx} \left(  \frac{1}{nx}- \ln \left( 1+ \frac{1}{nx} \right)  \right) = \frac{n}{nx^2+x}-\frac{1}{x^2} = 
\frac{nx^2-nx^2-x}{nx^4+x^3} = \frac{-x}{nx^4+x^3} < 0
\end{equation*}
Функция монотонно убывает на промежутке  $x \in (2, + \infty)$. Заметим, что в точке $x = 2$,  $n \left( \frac{1}{nx}- \ln \left( 1+ \frac{1}{nx} \right)  \right)$ определена и при $x \in [2, + \infty)$ имеет тот же характер монотонности, что и $x \in (2, + \infty)$

\begin{multline*}
0 \leq \max \limits_{x \in (2, + \infty) }  n \left( \frac{1}{nx}- \ln \left( 1+ \frac{1}{nx} \right)  \right) \leq \max \limits_{x \in [2, + \infty) }  n \left( \frac{1}{nx}- \ln \left( 1+ \frac{1}{nx} \right)  \right) =\\
=  n \left( \frac{1}{2n}- \ln \left( 1+ \frac{1}{2n} \right)  \right)
\end{multline*}
\begin{multline*}
0 \leq \lim_{n \to \infty} \left| f_n (x) - \phi(x)  \right| \leq \lim_{n \to \infty} n \left( \frac{1}{2n}- \ln \left( 1+ \frac{1}{2n} \right)  \right) =
 \frac{1}{2} - \lim_{n \to \infty} n \ln \left( 1+ \frac{1}{2n} \right)  = 0
\end{multline*}
Следовательно, при $x \in (2, + \infty)$ сходится \textbf{равномерно}.\\

\underline{Ответ:}  а) \textbf{сходится неравномерно}, б) \textbf{сходится равномерно}.


\newpage
\section{Равномерная сходимость ряда}
\subsection{а}
\begin{equation*}
\sum  \limits_{n = 1}^{\infty} \frac{\sin x \sin nx}{\sqrt{n + x}}
\end{equation*}
при $ x \in [0, + \infty)$.


\newpage
\section{Сумма функционального ряда}
\begin{equation*}
\frac{1}{2 \cdot 3} - \frac{x}{3 \cdot 4} + \frac{x^2}{4 \cdot 5} - ... + \frac{(-1)^n x^n}{(n+ 2) \cdot(n+ 3)} + ... = 
\sum  \limits_{n = 0}^{\infty} \frac{(-1)^n x^n}{(n+ 2) \cdot(n+ 3)}
\end{equation*}
Найдем область сходимости данного ряда. Применяя признак  д'Аламбера, получим
\begin{equation*}
\lim_{n \to \infty} \left| \frac{u_{n+1} (x)}{u_{n} (x)} \right| =
 \lim_{n \to \infty}  \left| \frac{(-1)^{n+1} x^{n+1}(n+ 2) (n+ 3)}{(n+ 3) (n+ 4) (-1)^n x^n} \right| = 
 \lim_{n \to \infty}  \left| - x \right| = \left|  x \right|  < 1
\end{equation*}
По признаку д'Аламбера ряд сходится при $ x \in (-1, 1)$. Дополнительно исследуем сходимость в точках $\pm 1$. Применим признак Гаусса
\begin{equation*}
 \left| \frac{u_{n+1} (\pm 1)}{u_{n} (\pm 1)} \right| = \frac{n+ 2}{n+ 4} = 1 - \frac{2}{n} + O\left(\frac{1}{n^2} \right)
\end{equation*}
В признаке Гаусса $\mu = -2 < -1$, следовательно при $x = \pm 1$ ряд сходится.\\
Область \textbf{сходимости} -- промежуток $[-1, 1]$.\\
Ряд сходится к функции $f(x)$. По теореме о дифференцировании степенного ряда на промежутке $ (-1, 1)$ этот ряд можно родиффиренцировать почленно. Следовательно,

\begin{equation*}
f'(x) = \sum  \limits_{n = 0}^{\infty} \frac{n (-1)^n x^{n-1}}{(n+ 2) \cdot(n+ 3)} = 
0 - \frac{1}{3 \cdot 4} + \frac{2x}{4 \cdot 5} - \frac{3x^2}{5 \cdot 6}+... + \frac{n (-1)^n x^{n-1}}{(n+ 2) \cdot(n+ 3)} +...
\end{equation*}
\end{document}










